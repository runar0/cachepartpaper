Chip multiprocessors (CMPs) commonly share the Last Level Cache (LLC) to improve resource utilization.
Thus, co-running processes compete for cache space which can result in significant and unpredictable performance loss.
The performance impact of this problem can be reduced by intelligently partitioning the LLC between processes, and a considerable amount of research that propose new LLC partitioning techniques exists.
Partitioning techniques tend to partially overlap, but this overlap can be hard to identify.
This problem is exacerbated by the sheer volume of techniques available.
Furthermore, researchers commonly compare their proposed technique to a limited number of competing schemes.
Thus, there is a real risk that researchers reinvent old ideas instead of identifying novel concepts.

In this work, we propose to classify LLC partitioning techniques along three axes.
The \emph{policy} defines the performance-related goal which partitioning is to achieve, the \emph{feedback mechanism} gathers information about the LLC requirements of the currently running processes and the \emph{partitioning mechanism} select and enforce partitions.
We apply this classification scheme to dynamic partitioning techniques with a miss minimizing policy and identify the amount of overlap.
In addition, we implement and compare the high-impact TADIP, DRRIP, UCP, PIPP and PriSM partitioning techniques.
UCP is the oldest of the compared techniques but still the top performer.
This indicates that more rigorous evaluation is necessary to ensure that new partitioning techniques are in fact an improvement over previous work.